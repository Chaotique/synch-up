The first plot shows Im and Re and the second Amplitude and Phase of the order parameter for the abovementioned model.\\
\begin{figure}[h!]\centering\includegraphics[scale = 0.3]{num/github/integratororder20150312-190034.pdf}\includegraphics[scale = 0.3]{num/github/integratorImRe20150312-190034.pdf}\caption{this plot was generated using integrator.py,
 integration with,  N\_osc =100, omg = randn, spread = 1, epsilon = 1.9, N\_time = 7000, tmax = 300.0}\end{figure}
integrator.py was the first attemt to see finitesize effects for gaussian distributed natural frequencies\\

\begin{figure}[h!]\centering\includegraphics[scale = 0.3]{num/github/integratororder20150313-110557.pdf}\includegraphics[scale = 0.3]{num/github/integratorImRe20150313-110557.pdf}\caption{this plot was generated using integrator.py, integration with N\_osc =100,  omg = randn, spread = 1
epsilon = 1.9, N\_time = 7000, tmax = 300.0, mean(R)~=0.65
}\end{figure}
three pics for N = 100 should e enough..
\begin{figure}[h!]\centering\includegraphics[scale = 0.3]{num/github/integratororder20150313-113455.pdf}\includegraphics[scale = 0.3]{num/github/integratorImRe20150313-113455.pdf}\caption{this plot was generated using integrator.py, integration with N\_osc =100,omg = randn, spread = 1,epsilon = 1.9,N\_time = 7000,tmax = 300.0, would have been good to save the full omg and y somewhere...
}\end{figure}
ok, one last realization:
\begin{figure}[h!]\centering\includegraphics[scale = 0.3]{num/github/integratororder20150313-114552.pdf}\includegraphics[scale = 0.3]{num/github/integratorImRe20150313-114552.pdf}\caption{this plot was generated using integrator.py, integration with N\_osc =100,omg = randn, spread = 1,epsilon = 1.9,N\_time = 7000,tmax = 300.0, }\end{figure}
now with fewer oscillators and much smaller coupling
\begin{figure}[h!]\centering\includegraphics[scale = 0.3]{num/github/integratororder20150313-114810.pdf}\includegraphics[scale = 0.3]{num/github/integratorImRe20150313-114810.pdf}\caption{this plot was generated using integrator.py, integration with N\_osc =25, omg = randn, spread = 1, epsilon = 1.0, N\_time = 7000, tmax = 300.0, well, lets choose $K$ a little bigger next time..}\end{figure}
bigger $K$
\begin{figure}[h!]\centering\includegraphics[scale = 0.3]{num/github/integratororder20150313-114949.pdf}\includegraphics[scale = 0.3]{num/github/integratorImRe20150313-114949.pdf}\caption{this plot was generated using integrator.py, integration with N\_osc =25, omg = randn, spread = 1, epsilon = 1.3, N\_time = 7000, tmax = 300.0, even bigger..?}\end{figure}
== $K$ of the examples with 100 oscis
\begin{figure}[h!]\centering\includegraphics[scale = 0.3]{num/github/integratororder20150313-115101.pdf}\includegraphics[scale = 0.3]{num/github/integratorImRe20150313-115101.pdf}\caption{this plot was generated using integrator.py, integration with N\_osc =25, omg = randn, spread = 1, epsilon = 1.9, N\_time = 7000, tmax = 300.0, too big..}\end{figure}
lets see...
\begin{figure}[h!]\centering\includegraphics[scale = 0.3]{num/github/integratororder20150313-115207.pdf}\includegraphics[scale = 0.3]{num/github/integratorImRe20150313-115207.pdf}\caption{this plot was generated using integrator.py, integration with N\_osc =25, omg = randn, spread = 1, epsilon = 1.5, N\_time = 7000, tmax = 300.0, winding number..}\end{figure}

maybe next time, check with even fewer N\_osc
\begin{figure}[h!]\centering\includegraphics[scale = 0.3]{num/github/integratororder20150313-115441.pdf}\includegraphics[scale = 0.3]{num/github/integratorImRe20150313-115441.pdf}\caption{this plot was generated using integrator.py, integration with N\_osc =25, omg = randn, spread = 1, epsilon = 1.5, N\_time = 7000, tmax = 300.0, funny: for rather repetitive behaviour, mean $R$ is much bigger than for mixed/ chaotic state}\end{figure}


\color{red}TODO:\color{black}: mean$R$ mit ausgeben lassen, ...\\ 

\begin{figure}[h!]\centering\includegraphics[scale = 0.3]{num/github/integratororder20150313-121023.pdf}\includegraphics[scale = 0.3]{num/github/integratorImRe20150313-121023.pdf}\caption{this plot was generated using integrator.py, integration with N\_osc =25, omg = randn, spread = 1, epsilon = 1.5, N\_time = 7000, tmax = 300.0, after deleting 8 realisations}\end{figure}
\begin{figure}[h!]\centering\includegraphics[scale = 0.3]{num/github/integratororder20150313-121322.pdf}\includegraphics[scale = 0.3]{num/github/integratorImRe20150313-121322.pdf}\caption{this plot was generated using integrator.py, integration with N\_osc =25, omg = randn, spread = 1, epsilon = 1.5, N\_time = 7000, tmax = 300.0, nice}\end{figure}
Zwischenbemerkung: die periodischen Realisierungen ($R$ und $\theta$ haben dieselbe Periode) sind selten, wahrscheinlich mit zunehmendem $N$ seltener. 

at least something\\ 
\begin{figure}[h!]\centering\includegraphics[scale = 0.3]{num/github/integratororder20150313-131709.pdf}\includegraphics[scale = 0.3]{num/github/integratorImRe20150313-131709.pdf}\caption{this plot was generated using integrator.py, integration with N\_osc =25, omg = randn, spread = 1, epsilon = 1.5, N\_time = 7000, tmax = 300.0, autocorrelation?}\end{figure}


